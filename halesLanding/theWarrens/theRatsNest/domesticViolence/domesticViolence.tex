% Preamble
\documentclass[11pt]{article}
\newcommand{\Z}{\mathbb{Z}}
\newcommand{\N}{\mathbb{N}}
\newcommand{\Q}{\mathbb{Q}}
\newcommand{\R}{\mathbb{R}}
\newcommand{\C}{\mathbb{C}}
\newcommand{\F}{\mathbb{F}}
\newcommand{\Gal}{\text{Gal}}
\newcommand{\script}[1]{\mathscr{#1}}
\newcommand{\partialby}[2]{\frac{\partial{#1}}{\partial{#2}}}
\newcommand{\partialbytwo}[2]{\frac{\partial^2{#1}}{\partial{#2}^2}}
\newcommand{\partialbyoneone}[3]{\frac{\partial^2{#1}}{\partial{#2} \partial{#3}}}
\newcommand{\mtx}[4] {\left[\begin{matrix} #1 & #2 \\ #3 & #4  \end{matrix}\right]}
\newcommand{\cmtx}[2]{\left[\begin{array}{c} #1 \\ #2 \end{array} \right]}
\newcommand{\rmtx}[2]{\left[\begin{array}{cc} #1 & #2 \end{array} \right]}
\newcommand{\abs}[1]{\left|#1\right|}
\newcommand{\norm}[1]{\|#1\|}
\newcommand{\lang}{\langle}
\newcommand{\rang}{\rangle}
\newcommand{\abrack}[1]{\lang #1 \rang}
\newcommand{\bra}[1]{\langle #1 |}
\newcommand{\ket}[1]{| #1 \rangle}
\newcommand{\braket}[2]{\langle #1 | #2 \rangle}
\newcommand{\floor}[1]{\lfloor #1 \rfloor}
\usepackage{amsmath, amsthm, amsfonts, amstext, amscd, amssymb, amsthm, fullpage, mathrsfs, color, verse, nopageno, tipa}
\usepackage[top=.5in, bottom=.5in, left=1in, right=1in]{geometry}
\usepackage{titlesec}

\setcounter{secnumdepth}{4}

\titleformat{\paragraph}
{\normalfont\normalsize\bfseries}{\theparagraph}{1em}{}
\titlespacing*{\paragraph}
{0pt}{3.25ex plus 1ex minus .2ex}{1.5ex plus .2ex}

\DeclareMathOperator{\tr}{Tr}
\newcommand*{\threeemdash}{\rule[0.5ex]{3em}{0.55pt}}
\newcommand*{\xdash}[1][3em]{\rule[0.5ex]{#1}{0.55pt}}
\makeatletter
\newcommand{\rmnum}[1]{\romannumeral #1}
\newcommand{\Rmnum}[1]{\expandafter\@slowromancap\romannumeral #1@}
\makeatother
% Theorem Styles
\newtheorem{theorem}{Theorem}[section]
\newtheorem{lemma}[theorem]{Lemma}
\newtheorem{proposition}[theorem]{Proposition}
\newtheorem{corollary}[theorem]{Corollary}
% Definition Styles
\theoremstyle{definition}
\newtheorem{definition}{Definition}[section]
\newtheorem{example}{Example}[section]
\theoremstyle{remark}
\newtheorem{remark}{Remark}
\usepackage[parfill]{parskip}
\newtheorem*{prop}{Proposition}

\begin{document}

% Heading
	\title{Domestic Violence}
	\date{\vspace{-24pt}}
	\author{Gabe Foster}
	\maketitle

\section{Domestic Violence}

\begin{description}

\item[Hooks:] 

\begin{itemize}

\item{Woman outside is crying for help for her child.}

\item{Guards try to take child away from woman, man in the Rat's Ass runs out to help.}

\end{itemize}

\item[NPCs:]
\begin{description}

\item[Beggar Woman:]

This unnamed woman frequents the streets outside the Rat's Ass, begging for money to hire a doctor to save her child.  This, however, is a guise. The child she holds is the body of whatever child has most recently been interred in the graveyard, and is quite dead.

\item[Seedy Man:]

This unnamed man spends his days drinking inside the Rat's Ass tavern. He uses this as a home base to observe the woman outside the establishment begging for coin, whom he protects and is in cahoots with.  At night, he skims coin from the deceased, and he and the woman have amassed an not insignificant trove within their hovel.

\item[Paulor, Glass Guard]

Paulor is assigned to the area of the Rat's Nest which contains the Rat's Ass, and has been following this woman for a while now.  He's noticed that the child she carries seems to change size every few days, and has noted as much in his journal.

\end{description}

\item{Locations:}

\begin{itemize}

\item{The Rat's Ass Tavern}

\item{Run Down House} The southernmost house in the district of the Rat's Ass which abuts the graveyard in Gods' Rest serves as a lair for these predators.

\end{itemize}

\end{description}

\end{document}