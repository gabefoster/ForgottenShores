% Preamble
\documentclass[11pt]{article}
\newcommand{\Z}{\mathbb{Z}}
\newcommand{\N}{\mathbb{N}}
\newcommand{\Q}{\mathbb{Q}}
\newcommand{\R}{\mathbb{R}}
\newcommand{\C}{\mathbb{C}}
\newcommand{\F}{\mathbb{F}}
\newcommand{\Gal}{\text{Gal}}
\newcommand{\script}[1]{\mathscr{#1}}
\newcommand{\partialby}[2]{\frac{\partial{#1}}{\partial{#2}}}
\newcommand{\partialbytwo}[2]{\frac{\partial^2{#1}}{\partial{#2}^2}}
\newcommand{\partialbyoneone}[3]{\frac{\partial^2{#1}}{\partial{#2} \partial{#3}}}
\newcommand{\mtx}[4] {\left[\begin{matrix} #1 & #2 \\ #3 & #4  \end{matrix}\right]}
\newcommand{\cmtx}[2]{\left[\begin{array}{c} #1 \\ #2 \end{array} \right]}
\newcommand{\rmtx}[2]{\left[\begin{array}{cc} #1 & #2 \end{array} \right]}
\newcommand{\abs}[1]{\left|#1\right|}
\newcommand{\norm}[1]{\|#1\|}
\newcommand{\lang}{\langle}
\newcommand{\rang}{\rangle}
\newcommand{\abrack}[1]{\lang #1 \rang}
\newcommand{\bra}[1]{\langle #1 |}
\newcommand{\ket}[1]{| #1 \rangle}
\newcommand{\braket}[2]{\langle #1 | #2 \rangle}
\newcommand{\floor}[1]{\lfloor #1 \rfloor}
\usepackage{amsmath, amsthm, amsfonts, amstext, amscd, amssymb, amsthm, fullpage, mathrsfs, color, verse, nopageno, tipa}
\usepackage[top=.5in, bottom=.5in, left=1in, right=1in]{geometry}
\usepackage{titlesec}

\setcounter{secnumdepth}{4}

\titleformat{\paragraph}
{\normalfont\normalsize\bfseries}{\theparagraph}{1em}{}
\titlespacing*{\paragraph}
{0pt}{3.25ex plus 1ex minus .2ex}{1.5ex plus .2ex}

\DeclareMathOperator{\tr}{Tr}
\newcommand*{\threeemdash}{\rule[0.5ex]{3em}{0.55pt}}
\newcommand*{\xdash}[1][3em]{\rule[0.5ex]{#1}{0.55pt}}
\makeatletter
\newcommand{\rmnum}[1]{\romannumeral #1}
\newcommand{\Rmnum}[1]{\expandafter\@slowromancap\romannumeral #1@}
\makeatother
% Theorem Styles
\newtheorem{theorem}{Theorem}[section]
\newtheorem{lemma}[theorem]{Lemma}
\newtheorem{proposition}[theorem]{Proposition}
\newtheorem{corollary}[theorem]{Corollary}
% Definition Styles
\theoremstyle{definition}
\newtheorem{definition}{Definition}[section]
\newtheorem{example}{Example}[section]
\theoremstyle{remark}
\newtheorem{remark}{Remark}
\usepackage[parfill]{parskip}
\newtheorem*{prop}{Proposition}

\begin{document}

% Heading
	\title{Empty Pockets}
	\date{\vspace{-24pt}}
	\author{Gabe Foster}
	\maketitle

\section{Empty Pockets}

\begin{description}

\item[Hooks:] 

\begin{itemize}

\item{Morning comes to the Rat's Ass tavern.  Hangovers are drunk off, and the time comes for debts to be paid.  Per the rules of the Rat's Ass, you pay what you carry.  As the PCs go to check their pockets for coin, they find them... empty?}

\end{itemize}

\item[NPCs:]
\begin{description}

\item[Kared:] Matron of the Rat's Ass Tavern, Kared appears to be a plain-faced woman with grey streaks running through her hair.  She keeps the bar at the tavern, and maintains the premises.  In truth, however, she is much more...

\item[Grub:] Grub serves as the bouncer for the Rat's Ass Tavern. He wears a key around his neck.

\item[Docksmaster Ellamin Highhold]

As part of his plots, the Docksmaster has been attempting to learn about the denizens of Hale's Landing. Attempting to garner information regarding Kared, who he does not yet fully fathom, he sends an invisible stalker to explore the Rat's Ass.  To further his ends, he steals from the residents and attempts to pin the theft on the Scalpers.

\item[Sadras, Scalper Agent]

Sadras is a Scalper agent and the contact of Mae Whitesmith, a Scalper double agent within the Docksmen.  Ellamin Highhold recently learned of this plot against his organization. As a result, he sent an invisible stalker with a kerchief of Sadras' (to frame the agent) to the Rat's Ass.

\end{description}

\item{Locations:}

\begin{itemize}

\item{The Rat's Ass Tavern}

\end{itemize}

\end{description}

\end{document}